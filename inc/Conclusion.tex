\section{Conclusion}

% \subsection{Exploring Emerging Challenges}

% Spider dataset extensions SParC and CoSQL are developed for contextual cross-domain semantical parsing and conversational dialog text-to-SQL systems.
% As a result, these fresh aspects provide new and significant issues for future research in this sector.

% \subsubsection{Conversational Text-to-SQL} \label{sec:conv}

% \subsubsection{SpaRC}

% \subsubsection{CoSQL Dataset}
% \input{inc/future/splash}

% \subsection{Conclusion and Summary of findings}

In this thesis, we discussed the state-of-the-art text-to-SQL solutions from a cross-domain perspective, providing a comprehensive overview of the current progress in the field. We demonstrated the effectiveness of pre-trained embeddings in improving schema linking and SQL structure accuracy through experimental results. We hope this study will shed light on the fundamental similarities as well as differences between older models and more recent approaches.

We also explored the dataset's impact on the text-to-SQL models' performance. We showed that the Spider dataset is a challenging benchmark for the text-to-SQL task. We also demonstrated the challenging SEOSS dataset and worked on some experiments with state-of-the-art models. Our comparison of different models for Text-to-SQL tasks shows that the PICARD + T5 model is a promising choice. However, the potential for even better results exists through fine-tuning the PICARD + T5 model. This process could lead to even more accurate results, but it would likely require access to high-end computing resources. These results demonstrate the importance of considering both model architecture and computational resources when evaluating the performance of NLP models.

Investigating new solutions and the need for more robust evaluation metrics now need to be addressed and further explored in future research. Additionally, with the growth of research in the transformer and language models field, new challenges, such as the Conversation-to-SQL task, have emerged and warrant further research directions.

In conclusion, text-to-SQL has witnessed significant progress over the past few years due to the development of cutting-edge datasets, models, and evaluation metrics. This field offers a wide variety of possibilities for ongoing research and technological advancement.

\clearpage

\section{Discussion and Future Directions}

The field of text-to-SQL is a rapidly growing area of research, with numerous systems and approaches proposed to generate SQL queries from natural language text. However, there are still several areas that require further exploration and improvement.

A promising avenue for further research is cross-domain text-to-SQL. Incorporating domain-specific knowledge into models trained on existing datasets would enable them to be more adaptable and applicable in different domains. Furthermore, this would facilitate their capacity to handle scenarios where the tables are corrupted or unavailable. In addition, advanced handling of user inputs that are different from the existing datasets and providing database administrators with the ability to manage database schemas and update content are key real-world applications of text-to-SQL. Additionally, multilingual text-to-SQL and creating a database interface for the disabled are noteworthy directions for future research.

Incorporating Text-to-SQL into a broader range of research areas, such as constructing a question-answering system for databases or a dialog system with knowledge from databases, could promote progress in the field. Investigating the interconnection between SQL and other logical forms, as well as generalized semantic parsing, would yield a more comprehensive comprehension of the topic and facilitate the development of more adjustable and generalizable systems.

Regarding more focused strategies, prompt learning could be utilized to improve the robustness of text-to-SQL, and existing text-to-SQL systems could be evaluated and compared to identify their advantages and disadvantages.

In sum, the domain of text-to-SQL has much potential for growth and progress, with numerous significant realistic applications and prospects for integration with related fields.