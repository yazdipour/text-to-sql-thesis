\section{Conclusion}

In this thesis, we delved into state-of-the-art text-to-SQL solutions from a cross-domain perspective, offering a comprehensive overview of the latest advancements in the field. We demonstrated the efficacy of pre-trained embeddings in enhancing schema linking and SQL structure accuracy through empirical results. This study aims to elucidate the fundamental similarities and differences between older models and contemporary approaches.
Additionally, we examined the influence of datasets on the performance of text-to-SQL models. We established that the Spider dataset serves as a challenging benchmark for text-to-SQL tasks. We also explored the demanding SEOSS dataset and conducted experiments with state-of-the-art models, including ChatGPT. Our comparison of various models for text-to-SQL tasks reveals that the PICARD + T5 model is a promising option. Nevertheless, the potential for improvement exists through fine-tuning the PICARD + T5 model, which could yield even more accurate results but may necessitate access to high-performance computing resources. These findings underscore the importance of considering both model architecture and computational resources when assessing the performance of NLP models.

Future research should address the exploration of new solutions and the development of more robust evaluation metrics. Moreover, with the expansion of research in the transformer and language models field, emerging challenges such as the Conversation-to-SQL task have surfaced, indicating the need for further investigation.
In conclusion, the text-to-SQL domain has experienced significant progress in recent years thanks to the creation of innovative datasets, models, and evaluation metrics. This area presents a wealth of opportunities for continued research and technological growth.

\clearpage

\section{Discussion and Future Directions}

The field of text-to-SQL is a rapidly growing area of research, with numerous systems and approaches proposed to generate SQL queries from natural language text. However, there are still several areas that require further exploration and improvement.

A promising avenue for further research is cross-domain text-to-SQL. Incorporating domain-specific knowledge into models trained on existing datasets would enable them to be more adaptable and applicable in different domains. Furthermore, this would facilitate their capacity to handle scenarios where the tables are corrupted or unavailable. In addition, advanced handling of user inputs that are different from the existing datasets and providing database administrators with the ability to manage database schemas and update content are key real-world applications of text-to-SQL. Additionally, multilingual text-to-SQL and creating a database interface for the disabled are noteworthy directions for future research.

Incorporating Text-to-SQL into a broader range of research areas, such as constructing a question-answering system for databases or a dialog system with knowledge from databases, could promote progress in the field. Investigating the interconnection between SQL and other logical forms, as well as generalized semantic parsing, would yield a more comprehensive comprehension of the topic and facilitate the development of more adjustable and generalizable systems.

Regarding more focused strategies, prompt learning could be utilized to improve the robustness of text-to-SQL, and existing text-to-SQL systems could be evaluated and compared to identify their advantages and disadvantages.

In sum, the domain of text-to-SQL has much potential for growth and progress, with numerous significant realistic applications and prospects for integration with related fields.