\section{Conclusion}
In this thesis, we provided an in-depth analysis of state-of-the-art text-to-SQL solutions from a cross-domain perspective, delivering a comprehensive overview of the latest advancements in the field. We demonstrated the efficacy of pre-trained embeddings in enhancing schema linking and SQL structure accuracy through empirical results. This study aimed to elucidate the fundamental similarities and differences between older models and contemporary approaches.

We also examined datasets' influence on text-to-SQL models' performance, highlighting the Spider dataset as a challenging benchmark for text-to-SQL tasks and investigating the demanding SEOSS dataset. We conducted experiments with state-of-the-art models, including ChatGPT, and compared various models for text-to-SQL tasks. Our findings revealed that the PICARD + T5 model is a promising option, but further improvement could be achieved through fine-tuning, which might necessitate access to high-performance computing resources. These findings underscore the importance of considering both model architecture and computational resources when assessing the performance of NLP models.

Our exploration of evaluation metrics emphasized the need for more robust measures to assess text-to-SQL systems. Additionally, as research in the transformer and language models field expands, emerging challenges such as the Conversation-to-SQL task have surfaced, indicating the need for further investigation.

In conclusion, the text-to-SQL domain has experienced significant progress in recent years due to the creation of innovative datasets, models, and evaluation metrics. This area presents a wealth of opportunities for continued research and technological growth, paving the way for more advanced, efficient, and versatile solutions to complex natural language processing tasks.

\clearpage
\thispagestyle{plain}
\section{Future Directions}

The field of text-to-SQL is a rapidly growing area of research, encompassing numerous systems and approaches designed to generate SQL queries from natural language text. Despite considerable advancements, several areas still warrant further exploration and improvement.

One promising direction for future research is cross-domain text-to-SQL. By incorporating domain-specific knowledge into models trained on existing datasets, these models can become more adaptable and applicable across different domains. This adaptability also enhances their capacity to handle scenarios where tables are corrupted or unavailable. Furthermore, addressing real-world applications, such as advanced handling of user inputs not present in existing datasets and providing database administrators with tools to manage database schemas and update content, is essential for the practical implementation of text-to-SQL systems. Multilingual text-to-SQL and developing database interfaces for individuals with disabilities are essential areas for future research.

The growth of \ac{LLM} has significantly impacted various research fields, including text-to-SQL. Integrating text-to-SQL tasks into LLMs can lead to the development of more versatile question-answering systems for databases and dialogue systems with database-driven knowledge. Additionally, exploring the interconnection between SQL and other logical forms, as well as generalized semantic parsing, will contribute to a more comprehensive understanding of the subject, facilitating the creation of more adaptable and generalizable systems.
Prompt learning, an emerging strategy within the LLM landscape, can be employed to enhance the robustness of text-to-SQL systems. This approach involves training models to respond to specific prompts or questions, thereby improving their performance in generating accurate SQL queries from natural language text. 

In conclusion, the domain of text-to-SQL holds immense potential for growth and progress, with numerous significant practical applications and opportunities for integration with related fields, particularly in light of the ongoing advancements in large language models.