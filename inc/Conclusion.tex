\section{Conclusion}

\subsection{Exploring Emerging Challenges}

Spider dataset extensions SParC and CoSQL are developed for contextual cross-domain semantical parsing and conversational dialog text-to-SQL systems.
As a result, these fresh aspects provide new and significant issues for future research in this sector.

\subsubsection{Conversational Text-to-SQL} \label{sec:conv}

\subsubsection{SpaRC}

\subsubsection{CoSQL Dataset}
\input{inc/future/splash}

\subsection{Conclusion and Summary of findings}

In this thesis, we discussed the state-of-the-art text-to-SQL solutions from a cross-domain perspective, providing a comprehensive overview of the current progress in the field. We demonstrated the effectiveness of pre-trained embeddings in improving schema linking and SQL structure accuracy through experimental results. We hope this study will shed light on the key similarities as well as differences between older models and more recent approaches.

Investigating new solutions and the need for more robust evaluation metrics now need to be addressed and further explored in future research. Additionally, with the growth of research in the transformer and language models field, new challenges, such as the Conversation-to-SQL task\ref{sec:conv}, have emerged and warrant further research directions.

In conclusion, text-to-SQL has witnessed significant progress over the past few years due to the development of cutting-edge datasets, models, and evaluation metrics. This field offers a wide variety of possibilities for ongoing research and technological advancement.