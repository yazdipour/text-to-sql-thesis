\section*{Abstract}

This thesis comprehensively examines the latest advancements in text-to-SQL solutions, with a specific focus on the cross-domain perspective. We delve into the effectiveness of pre-trained embeddings in improving schema linking and SQL structure accuracy, concluding empirical data. Furthermore, we explore the distinctions and commonalities between traditional and modern approaches in the field.
Our analysis extends to the impact of datasets on the performance of text-to-SQL models, with the Spider and SEOSS datasets serving as our primary subjects of investigation. We assess state-of-the-art models like GPT-4 and others within text-to-SQL tasks, identifying the PICARD-T5 model as a potential candidate for enhancement through fine-tuning, subject to the availability of high-performance computing resources.
The study also underscores the necessity for more robust evaluation metrics for text-to-SQL systems and identifies emerging challenges, such as the Conversation-to-SQL task, signifying avenues for future research. In summary, the thesis conveys the significant strides made in the text-to-SQL field, facilitated by introducing innovative datasets, models, and evaluation metrics, while highlighting the potential for ongoing research and development in this promising domain.