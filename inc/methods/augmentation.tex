\subsection{Data Augmentation}
\label{sec:augmentation}

Data augmentation has emerged as a valuable method to enhance the performance of text-to-SQL models by helping them tackle complex or previously unseen questions (Zhong et al., 2020b; Wang et al., 2021b), achieve state-of-the-art results with less supervised data (Guo et al., 2018), and attain robustness towards different question types (Radhakrishnan et al., 2020).

Common data augmentation strategies include paraphrasing questions and using pre-defined templates to increase data diversity. Iyer et al. (2017) utilized the Paraphrase Database (PPDB) (Ganitkevitch et al., 2013) to generate paraphrased training questions. Additionally, researchers have used neural models to create natural utterances for sampled SQL queries, thus expanding the available data. For instance, Li et al. (2020a) fine-tuned the pre-trained T5 model (Raffel et al., 2019) on WikiSQL. They used the SQL query as input to predict natural utterances and then synthesized SQL queries from WikiSQL tables to generate corresponding natural utterances with the tuned model.

The quality of augmented data is crucial, as low-quality data can negatively impact model performance (Wu et al., 2021). Various techniques have been employed to improve augmented data quality. Zhong et al. (2020b) used an utterance generator to produce natural utterances and a semantic parser to convert these utterances into SQL queries, filtering out low-quality data by retaining only instances with generated SQL queries that matched the sampled ones. Wu et al. (2021) adopted a hierarchical SQL-to-question generation process to obtain high-quality data, decomposing SQL queries into clauses, translating each clause into a sub-question, and combining the sub-questions into a complete question.

To further diversify augmented data, Guo et al. (2018) added a latent variable to their SQL-to-text model to promote question diversity. Radhakrishnan et al. (2020) improved the WikiSQL dataset by simplifying and compressing questions to mimic the colloquial query behavior of end-users. Wang et al. (2021b) used a probabilistic context-free grammar (PCFG) to explicitly model the composition of SQL queries, which fostered the sampling of compositional SQL queries. These data augmentation techniques collectively contribute to the improvement of text-to-SQL models, enabling them to better handle a wider range of questions and adapt to previously unseen data.