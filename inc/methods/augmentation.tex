\subsection{Data Augmentation}
\label{sec:augmentation}

Data augmentation has proven to be an effective technique for improving the performance of text-to-SQL models, allowing them to address more complex or novel questions (Zhong et al. \cite{zhong_semantic_2020} and Wang et al. \cite{wang_rat_sql_2021}), achieve cutting-edge results with less supervised data, and enhance their adaptability to various question types (Radhakrishnan et al. \cite{DBLP:journals/corr/abs-2010-09927}).

Typical data augmentation approaches involve rephrasing questions and employing pre-established templates to boost data variety. Iyer et al. \cite{iyer-etal-2017-learning} made use of the Paraphrase Database (PPDB) (Ganitkevitch et al. \cite{ganitkevitch-etal-2013-ppdb}) to create rephrased training questions. Moreover, researchers have utilized neural models to generate natural-sounding expressions for sampled SQL queries, thus broadening the available data pool. For example, Li et al. \cite{raffel_exploring_2020} fine-tuned the pre-trained Raffel T5 model \cite{raffel_exploring_2020} on WikiSQL, using the SQL query to predict natural expressions and subsequently synthesizing SQL queries from WikiSQL tables to produce corresponding natural expressions with the refined model.

The quality of the augmented data is essential, as poor-quality data can adversely affect the performance of the model\cite{DBLP:journals/corr/abs-2009-13845}. Numerous methods have been applied to enhance the quality of augmented data. Zhong et al. \cite{zhong_semantic_2020} employed an utterance generator to create natural expressions and a semantic parser to convert these expressions into SQL queries. They filtered out insufficient data by retaining only instances where generated SQL queries matched the sampled ones. Wu et al. \cite{DBLP:journals/corr/abs-2009-13845} implemented a hierarchical SQL-to-question generation process to obtain high-quality data, breaking down SQL queries into clauses, translating each clause into a sub-question, and merging the sub-questions to form a comprehensive question.

To further diversify augmented data and encourage question variety, Guo et al. \cite{DBLP:journals/corr/abs-1905-08205} incorporated a latent variable into their SQL-to-text model. Wang et al. utilized a \ac{PCFG} to explicitly model the composition of SQL queries \cite{yang-etal-2021-pcfgs}, which facilitated the sampling of compound SQL queries. These data augmentation methods collectively contribute to the enhancement of text-to-SQL models, allowing them to more effectively handle a broader range of questions and adapt to previously unencountered data.