\subsection{Early Approaches}

Early approaches to Text-to-SQL focused on rule-based methods and template-based methods. These approaches relied on predefined templates and a set of predefined rules to generate SQL queries. These methods were based on the idea that a fixed set of templates and rules could be used to generate SQL queries for a wide range of natural language inputs. However, these methods were limited by their reliance on predefined templates and were not able to handle a wide range of natural language inputs.

\subsubsection{Rule-based methods}

In the case of rule-based methods, a set of predefined rules were used to map the natural language input to the corresponding SQL query. These rules were based on predefined grammar and were used to identify the SQL constructs present in the input text. These methods were able to generate simple SQL queries, but they were not able to handle more complex queries or handle variations in natural language inputs.

Early research in Text-to-SQL includes work by researchers such as Warren and Pereira in 1982\cite{Warren1982AnEE}, who proposed a rule-based method for generating SQL queries from natural language text. Their system used a set of predefined rules to map natural language constructs to SQL constructs and was able to generate simple SQL queries. Another example of a rule-based method is the work by Zelle and Mooney, who proposed CHILL parser\cite{Zelle1996LearningTP}, a system that used a predefined grammar to identify the SQL constructs present in the input text and generate the corresponding SQL query. However, these rule-based methods were limited by their reliance on predefined templates and grammar rules, making them incapable of handling complex natural language inputs.

\subsubsection{Template-based methods}

Template-based methods, on the other hand, relied on predefined templates to generate SQL queries. These templates were based on a predefined set of SQL constructs and were used to map the natural language input to the corresponding SQL query. These methods could handle a limited set of natural language inputs, but they could not handle variations in the input or generate more complex queries. One of the very first systems that used predefined templates to map natural language inputs to SQL queries was able to handle a limited set of natural language inputs. The system was called LUNAR and was developed by Woods in 1972 \cite{lunar}.
In summary, early approaches to Text-to-SQL were limited by their reliance on predefined templates and rules, which made them unable to handle a wide range of natural language inputs and generate complex SQL queries. The rule-based and template-based methods were two of the most common early approaches used in Text-to-SQL, each with its strengths and limitations.
