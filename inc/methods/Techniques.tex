\subsection{Learning Techniques}

The advancement of Text-to-SQL research has been driven by various learning techniques that address specific challenges in the field. We provide a comprehensive overview of these learning techniques, focusing on both fully supervised and weakly supervised methods.

\subsubsection{Fully Supervised Learning Techniques}

Fully supervised learning approaches depend on labeled data to train models. We explore various cutting-edge methods proposed to enhance Text-to-SQL generation.

\subsubsubsection{Active Learning}

Active learning aims to reduce the labeled data needed for training by selectively identifying the most informative examples. Ni et al.\cite{ni2019merging} developed an active learning framework that uses uncertainty estimation to pinpoint samples that would gain the most from human annotations.

\subsubsubsection{Interactive/Imitation Learning}

Interactive or imitation learning concentrates on learning from demonstrations, with a model trying to replicate expert behavior. Yao et al. \cite{yao-etal-2019-model} presented an interactive learning method that integrates user feedback to improve the model's comprehension of intricate SQL queries.

\subsubsubsection{Meta-learning}

Meta-learning entails training models to learn how to learn effectively. Huang et al. \cite{huang-etal-2018-natural} suggested a meta-learning strategy for Text-to-SQL tasks, enabling the model to swiftly adapt to new tasks or domains with limited labeled data.

\subsubsubsection{Multi-task Learning}

Multi-task learning consists of training a single model on multiple related tasks concurrently. Chang et al. \cite{chen2021leveraging} investigated a multi-task learning framework for Text-to-SQL generation, illustrating that sharing information across tasks can result in enhanced performance.

\subsubsection{Weakly Supervised Learning Techniques}

Weakly supervised learning approaches employ weak or noisy labels for training, often diminishing the need for extensive human annotation.

\subsubsubsection{Reinforcement Learning}

Reinforcement learning focuses on learning through trial and error, with models receiving feedback via rewards or penalties. Seq2SQL Zhong et al. \cite{zhong_seq2sql_2017} applied reinforcement learning to Text-to-SQL generation, demonstrating that such a method can effectively learn from weak supervision, for instance, it allowed Seq2SQL to understand different orders of WHERE clauses in a query.

\subsubsubsection{Meta-learning and Bayesian Optimization}

Agarwal et al. \cite{pmlr-v97-agarwal19e} combined meta-learning and Bayesian optimization for weakly supervised Text-to-SQL tasks. This technique enables models to adapt more efficiently to new tasks while taking advantage of limited supervision, ultimately reducing the necessity for large amounts of labeled data.