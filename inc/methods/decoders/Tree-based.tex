\subsubsection{Tree-based}

In the realm of text-to-SQL research, tree-based decoders have emerged as a popular approach for generating logical forms or abstract syntax trees (ASTs) from input text. Two key papers in this area are Dong and Lapata (2016) \cite{dong-lapata-2016-language} with their Seq2Tree model and Yin and Neubig (2017) \cite{yin-neubig-2017-syntactic} with their Seq2AST model. While these works do not specifically focus on text-to-SQL datasets, they inspire the development of tree-based decoding methods within the text-to-SQL context, such as SyntaxSQLNet by Yu et al. (2018b) \cite{DBLP:journals/corr/abs-1810-05237}.

The Seq2Tree model by Dong and Lapata (2016) \cite{dong-lapata-2016-language} creates logical forms through a top-down approach, where components of the sub-tree are generated based on their parent nodes, independently from the input question. This model learns the syntax of the logical forms implicitly, without any explicit guidance. On the other hand, the Seq2AST model by Yin and Neubig (2017)  \cite{yin-neubig-2017-syntactic} explicitly integrates syntax into the generation process through the use of an AST. This approach decodes the target programming language by constructing an AST that adheres to the language's syntax rules.

Taking inspiration from these approaches, SyntaxSQLNet by Yu et al. (2018b) \cite{DBLP:journals/corr/abs-1810-05237} adapts the tree-based decoding method to SQL syntax. This model employs a recursive structure that calls various modules to predict different SQL components, ultimately generating a valid SQL query. The model's tree-based decoding technique is tailored to the SQL language and facilitates a more structured prediction process.

In summary, tree-based decoders in text-to-SQL research offer a structured way to generate logical forms or ASTs, making use of either implicit or explicit syntax learning. By adapting these methods to the specific requirements of SQL syntax, researchers can develop more effective models for translating natural language questions into SQL queries.