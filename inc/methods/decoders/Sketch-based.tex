\subsubsection{Sketch-based}

Sketch-based decoders have gained attention in text-to-SQL research as an approach that simplifies the generation of SQL queries by leveraging predefined query structures, or "sketches." These sketches follow SQL grammar and allow the model to focus on filling in the slots rather than predicting the output grammar and content simultaneously.

SQLNet by Xu et al. (2017) \cite{xu_sqlnet_2017} is an example of a sketch-based model that aligns with SQL grammar. The sketch captures dependencies between predictions, which means that the prediction for each slot is conditioned only on the slots it depends on. This approach effectively avoids issues arising from equivalent serializations of the same SQL query.

Dong and Lapata (2018) \cite{dong-lapata-2018-coarse} further refine the sketch-based approach by decomposing the decoding process into two stages. The first decoder predicts a rough sketch, while the second decoder fills in the low-level details based on the input question and the sketch. This coarse-to-fine decoding has been adopted in other works, such as IRNet by Guo et al. (2019) \cite{DBLP:journals/corr/abs-1905-08205}.

To handle complex SQL queries with nested structures, RYANSQL by Choi et al. (2021) \cite{10.1162/coli_a_00403} introduces a recursive method for generating SELECT statements. This model employs sketch-based slot filling for each of the SELECT statements, enabling the generation of more intricate queries.

In summary, sketch-based decoders simplify the text-to-SQL generation process by providing predefined query structures that follow SQL grammar. This approach enables models to focus on filling in content slots, captures dependencies between predictions, and allows for the handling of complex queries with nested structures. By decomposing the decoding process into multiple stages, sketch-based decoders can efficiently translate natural language questions into accurate SQL queries.