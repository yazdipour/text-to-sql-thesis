\subsubsection{Intermediate Representations}

Intermediate representations (IRs) are employed in Text-to-SQL research to bridge the gap between natural language and SQL queries. By using IRs, researchers can simplify and abstract SQL queries, making it easier for models to learn and generate an accurate output.

IncSQL by Shi et al. (2018)\cite{shi2018incsql} is one such approach that defines actions for different SQL components, allowing the decoder to decode these actions instead of raw SQL queries. This method reduces the complexity of the decoding process and can improve the overall performance of the model.

IRNet by Guo et al. (2019) \cite{DBLP:journals/corr/abs-1905-08205} introduces SemQL, an intermediate representation for SQL queries designed to cover most of the challenging Spider benchmark. SemQL simplifies SQL queries by removing the JOIN ON, FROM, and GROUP BY clauses and merging the HAVING and WHERE clauses. ValueNet by Brunner and Stockinger (2021) \cite{brunner2021valuenet} builds upon SemQL by introducing SemQL 2.0, which extends the original representation to include value representation. Additionally, NatSQL by Gan et al. (2021c) \cite{gan-etal-2021-natural-sql} modifies SemQL by removing set operators, such as INTERSECT, which combine the results of two or more SELECT statements.

Suhr et al. (2020) \cite{semql} implement SemQL as a mapping from SQL to a representation with an under-specified FROM clause, which they call SQLUF. Rubin and Berant (2021) employ a relational algebra augmented with SQL operators as intermediate representations, offering another approach to simplifying SQL queries.

However, one of the main challenges with intermediate representations is that they are typically designed for specific datasets and cannot be easily adapted to others. To address this issue, Herzig et al. (2021) \cite{herzig2021unlocking} propose a more generalized intermediate representation by omitting tokens in the SQL query that do not align with any phrase in the natural language utterance.

The success of intermediate representations in Text-to-SQL tasks has inspired researchers to explore their use in other executable language domains, such as SPARQL for database systems. Works by Saparina and Osokin (2021) \cite{saparina-osokin-2021-sparqling} investigate the potential of intermediate representations for SPARQL queries.

In conclusion, intermediate representations play an essential role in Text-to-SQL research by simplifying and abstracting SQL queries, making it easier for models to learn and generate an accurate output. The exploration of various intermediate representation techniques continues to improve the performance of Text-to-SQL models and inspire advancements in other related domains.