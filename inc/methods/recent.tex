\subsection{Recent Approaches}

Recent approaches to Text-to-SQL have focused on using neural networks and machine learning techniques to generate SQL queries. These methods use large amounts of training data to learn the relationship between natural language and SQL and can generate SQL queries for a wide range of inputs. These methods can handle a wide range of natural language inputs and are not limited by predefined templates or rules. Additionally, recent approaches leverage pre-trained models such as \ac{BERT} \cite{devlin-etal-2019-bert}, GPT \cite{radford2019language}, and T5 \cite{raffel_exploring_2020}, which have been pre-trained on a large corpus of text, to fine-tune text-to-SQL tasks, which enables them to understand the natural language inputs better and generate more accurate SQL queries.

One favored strategy is using encoder-decoder architecture, which uses an encoder to encode the natural language input and a decoder to generate the corresponding SQL query. The encoder is a pre-trained language model such as BERT, which is fine-tuned on the task of text-to-SQL, and the decoder is a neural network that generates the SQL query. This architecture effectively generates accurate SQL queries for various natural language inputs.

Another recent approach is using reinforcement learning to generate SQL queries, where a neural network generates a sequence of SQL tokens and is trained using a reward signal based on the quality of the generated query. This approach is adequate for generating more complex SQL queries and handling variations in natural language inputs.


% The Transformer architecture, presented in the paper "Attention Is All You Need" by Vaswani in 2017, is a neural network architecture that uses self-attention mechanisms to process data sequences, such as natural language text.

% One of the key advantages of the Transformer architecture is its ability to handle long-term dependencies in data sequences, making it well-suited for tasks such as natural language understanding and text generation. This has led to the development of pre-trained Transformer models, such as BERT, GPT-2, and T5, that have been trained on a large corpus of text and can be fine-tuned on specific tasks such as Text-to-SQL.

The Transformer architecture \cite{https://doi.org/10.48550/arxiv.1706.03762} has considerably impacted natural language processing and machine learning, especially in the Text-to-SQL area. Pre-trained Transformer models like BERT enhance performance through their deep understanding of natural language, enabling accurate interpretation of input text and precise SQL query generation. Furthermore, their capacity to manage long-term dependencies in data sequences has been pivotal in generating correct SQL queries from diverse natural language inputs.

Therefore, new Text-to-SQL approaches employ neural networks and machine learning techniques, such as encoder-decoder architecture and reinforcement learning. These methods use large training datasets and pre-trained models like BERT to generate accurate SQL queries from an extensive range of natural language inputs.