\subsection{Decoding}

Decoders \cite{cho-etal-2014-learning} form an integral part of sequence-to-sequence models in natural language processing tasks, and they are constructed as a multi-layered architecture of recurrent elements, such as Long Short-Term Memory (LSTM) units, Gated Recurrent Units (GRUs), or other analogous structures. The primary responsibility of a decoder is to generate an output sequence by predicting an output, denoted as y, for each time step. This output sequence can be a series of words, phrases, or even entire sentences, depending on the specific problem being addressed.

At each time step, the current recurrent unit within the decoder receives a hidden state from the preceding recurrent unit. This hidden state encapsulates the information gathered up to that point and serves as a vital input for the current recurrent unit to make an informed prediction. Moreover, decoders can also incorporate attention mechanisms to help focus on the most relevant parts of the input sequence when generating the output. This is particularly useful in tasks that require the decoder to selectively attend to different input elements during the decoding process.

Decoders are commonly employed in a wide range of natural language processing applications \cite{kumar2022deep}, including but not limited to, machine translation, text summarization, question-answering systems, and dialogue generation. In question-answering tasks, for instance, the output sequence generated by the decoder is often a collection of words.

Numerous approaches have been suggested to enhance the decoding process for more precise and efficient SQL generation, ultimately bridging the divide between natural language and SQL query formulation. As illustrated in the table below, we have classified these techniques into five primary categories, along with additional methodologies\cite{deng2022recent}.

\begin{table}
    \centering
    \begin{tabular}{|c|c|c|c|}
        \hline
        \rowcolor{Gray}
        \textbf{Methods}                                           & \textbf{Adopted by} & \textbf{Applied datasets} & \textbf{Addressed challenges}                                                            \\
        \hline
        \multirow{3}{*}{Tree-based}                                & Seq2Tree            & -                         & \multirow{3}{*}{Hierarchical decoding}                                                   \\
                                                                   & Seq2AST             & -                         &                                                                                          \\
                                                                   & SyntaxSQLNet        & Spider                    &                                                                                          \\
        \hline
        \multirow{4}{*}{Sketch-based}                              & SQLNet              & WikiSQL                   & \multirow{4}{*}{Hierarchical decoding}                                                   \\
                                                                   & Coarse2Fine         & WikiSQL                   &                                                                                          \\
                                                                   & IRNet               & Spider                    &                                                                                          \\
                                                                   & RYANSQL             & Spider                    &                                                                                          \\
        \hline
        Bottom-up                                                  & SmBop               & Spider                    & Hierarchical decoding                                                                    \\
        \hline
        \multirow{2}{*}{Self-Attention}                            & Seq2Tree            & -                         & \multirow{2}{*}{ Synthesizing information}                                               \\
                                                                   & Seq2SQL             & WikiSQL                   &                                                                                          \\
        \hline
        Bi-attention                                               & BiSQL               & Spider                    & Synthesizing information                                                                 \\
        \hline
        \parbox{3cm}{Relation-aware Self-attention}                & DuoRAT              & Spider                    & Synthesizing information                                                                 \\
        \hline
        \multirow{3}{*}{Copy Mechanism}                            & Seq2AST             & -                         & \multirow{3}{*}{ Synthesizing information}                                               \\
                                                                   & Seq2SQL             & WikiSQL                   &                                                                                          \\
                                                                   & SeqGenSQL           & WikiSQL                   &                                                                                          \\
        \hline
        \multirow{3}{*}{\parbox{3cm}{Intermediate Representation}} & IncSQL              & WikiSQL                   & \multirow{3}{*}{{\parbox{5cm}{Bridging the gap between natural language and SQL query}}} \\
                                                                   & IRNet               & WikiSQL                   &                                                                                          \\
                                                                   & ValueNet            & Spider                    &                                                                                          \\
        \hline
        Constrained decoding                                       & PICARD              & Spider                    & Fine-grained decoding                                                                    \\
        % \hline
        % Execution-guided                                           & SQLova              & WikiSQL                   & Fine-grained decoding                                                                    \\
        % \hline
        % Separate submodule                                         & SQLNet              & WikiSQL                   & Easier decoding                                                                          \\
        % \hline
        % BPE                                                        & BPESQL              & Advising, ATIS
        %    & Easier decoding
        % \\
        \hline
    \end{tabular}
    \caption{Methods used for decoding in text-to-SQL \cite{deng2022recent}}
    \label{tab:decoders}
\end{table}

\clearpage
\subsubsection{Tree-based}

\subsubsection{Sketch-based}

Sketch-based decoders have gained attention in text-to-SQL research as an approach that simplifies the generation of SQL queries by leveraging predefined query structures, or "sketches." These sketches follow SQL grammar and allow the model to focus on filling in the slots rather than predicting the output grammar and content simultaneously.

SQLNet by Xu et al.  \cite{xu_sqlnet_2017} is an example of a sketch-based model that aligns with SQL grammar. The sketch captures dependencies between predictions, which means that the prediction for each slot is conditioned only on the slots it depends on. This approach effectively avoids issues arising from equivalent serializations of the same SQL query.

Dong and Lapata  \cite{dong-lapata-2018-coarse} further refine the sketch-based approach by decomposing the decoding process into two stages. The first decoder predicts a rough sketch, while the second decoder fills in the low-level details based on the input question and the sketch. This coarse-to-fine decoding has been adopted in other works, such as IRNet by Guo et al.  \cite{DBLP:journals/corr/abs-1905-08205}.

To handle complex SQL queries with nested structures, RYANSQL by Choi et al.  \cite{10.1162/coli_a_00403} introduces a recursive method for generating SELECT statements. This model employs sketch-based slot filling for each of the SELECT statements, enabling the generation of more intricate queries.

In summary, sketch-based decoders simplify the text-to-SQL generation process by providing predefined query structures that follow SQL grammar. This approach enables models to focus on filling in content slots, captures dependencies between predictions, and allows for the handling of complex queries with nested structures. By decomposing the decoding process into multiple stages, sketch-based decoders can efficiently translate natural language questions into accurate SQL queries.
\subsubsection{Bottom-up}

\subsubsection{Attention Mechanism}

Attention mechanism decoders play a critical role in integrating encoder-side information during the decoding process. By computing attention scores and multiplying them with hidden vectors from the encoder, a context vector is generated, which is then used to produce an output token.

Various attention structures have been employed to enhance the decoder's performance and effectively propagate the information encoded from questions and database schemas. One such example is SQLNet (Xu et al., 2017) \cite{xu_sqlnet_2017}, which introduces the concept of column attention. This technique involves using hidden states from columns and multiplying them by embeddings for the question to calculate attention scores for a given column. The attention scores are then used to help the model focus on relevant columns when generating the SQL query.

Another approach, proposed by Guo and Gao (2018) \cite{guo2020content}, incorporates bi-attention over a question and column names for SQL component selection. This method enables the model to simultaneously attend to both the question and column names, which can improve the model's ability to identify and select relevant SQL components.

Wang et al. (2019) \cite{wang-etal-2019-learning} adopt a structured attention mechanism \cite{kim2017structured} that computes marginal probabilities to fill in the slots of their generated abstract SQL queries. This approach allows the model to better capture the structure of SQL queries and enhances the overall generation process.

DuoRAT \cite{scholak-etal-2021-duorat} implements a relation-aware self-attention mechanism in both its encoder and decoder components. This attention mechanism accounts for relationships between different elements within the input data, thus improving the model's ability to comprehend and generate accurate SQL queries.

Other works, such as those by Scholak et al. PICARD (2021b) \cite{Scholak2021:PICARD} and UnifiedSKG by Xie et al. (2022) \cite{xie2022unifiedskg}, use sequence-to-sequence transformer-based models or decoder-only transformer-based models that incorporate the self-attention mechanism by default. The self-attention mechanism allows the model to weigh the significance of each input token concerning other tokens in the sequence, which can enhance the quality and coherence of the generated output.

In summary, attention mechanism decoders have been an essential aspect of Text-to-SQL research, with various structures designed to improve the propagation of information and the generation of accurate SQL queries. By continuously refining and adapting these attention mechanisms, researchers aim to further enhance the performance of Text-to-SQL models.
\subsubsection{Copy Mechanism}

\subsubsection{Intermediate Representations}

Intermediate representations (IRs) are employed in Text-to-SQL research to bridge the gap between natural language and SQL queries. By using IRs, researchers can simplify and abstract SQL queries, making it easier for models to learn and generate an accurate output.

IncSQL by Shi et al. (2018)\cite{shi2018incsql} is one such approach that defines actions for different SQL components, allowing the decoder to decode these actions instead of raw SQL queries. This method reduces the complexity of the decoding process and can improve the overall performance of the model.

IRNet by Guo et al. (2019) \cite{DBLP:journals/corr/abs-1905-08205} introduces SemQL, an intermediate representation for SQL queries designed to cover most of the challenging Spider benchmark. SemQL simplifies SQL queries by removing the JOIN ON, FROM, and GROUP BY clauses and merging the HAVING and WHERE clauses. ValueNet by Brunner and Stockinger (2021) \cite{brunner2021valuenet} builds upon SemQL by introducing SemQL 2.0, which extends the original representation to include value representation. Additionally, NatSQL by Gan et al. (2021c) \cite{gan-etal-2021-natural-sql} modifies SemQL by removing set operators, such as INTERSECT, which combine the results of two or more SELECT statements.

Suhr et al. (2020) \cite{semql} implement SemQL as a mapping from SQL to a representation with an under-specified FROM clause, which they call SQLUF. Rubin and Berant (2021) employ a relational algebra augmented with SQL operators as intermediate representations, offering another approach to simplifying SQL queries.

However, one of the main challenges with intermediate representations is that they are typically designed for specific datasets and cannot be easily adapted to others. To address this issue, Herzig et al. (2021) \cite{herzig2021unlocking} propose a more generalized intermediate representation by omitting tokens in the SQL query that do not align with any phrase in the natural language utterance.

The success of intermediate representations in Text-to-SQL tasks has inspired researchers to explore their use in other executable language domains, such as SPARQL for database systems. Works by Saparina and Osokin (2021) \cite{saparina-osokin-2021-sparqling} investigate the potential of intermediate representations for SPARQL queries.

In conclusion, intermediate representations play an essential role in Text-to-SQL research by simplifying and abstracting SQL queries, making it easier for models to learn and generate an accurate output. The exploration of various intermediate representation techniques continues to improve the performance of Text-to-SQL models and inspire advancements in other related domains.
\subsubsection{Constrained decoding}

Constrained decoding methods are employed in natural language processing tasks, such as text-to-SQL, to improve the quality of generated outputs by imposing certain constraints or utilizing auxiliary models during the decoding process. These methods aim to prevent the generation of invalid tokens, exclude non-executable partial SQL queries, or facilitate the generation of complete SQL queries.

PICARD by Scholak et al., \cite{Scholak2021:PICARD} is an example of a method that sets constraints on the decoder to avoid generating invalid tokens. Other methods, such as those proposed by Wang et al. \cite{wang2018robust} and Hwang et al. \cite{DBLP:journals/corr/abs-1902-01069}, adopt an execution-guided decoding mechanism that eliminates non-executable partial SQL queries from the output candidates.

Some approaches, like Global-GNN, Bogin et al. \cite{bogin-etal-2019-global}, use separately trained discriminative models to rerank the top-K SQL queries in the decoder's output beam. This technique allows the model to reason about complete SQL queries rather than considering each word and database schema in isolation.

Chen et al. \cite{chen-etal-2020-tale} employ a gating mechanism to select between the output sequence encoded for the question and the output sequence from the previous decoding steps at each step for SQL generation. This approach helps in generating more accurate and coherent SQL queries.

Müller and Vlachos \cite{müller2019bytepair} draw inspiration from machine translation and apply \ac{BPE} (Sennrich et al.\cite{sennrich-etal-2016-neural}) to compress SQL queries into shorter sequences, guided by AST. This technique reduces the difficulties in SQL generation, leading to improved performance in text-to-SQL tasks.


% Schema linking is a component of text-to-SQL models that helps map natural language phrases to elements of a database schema.
% Skeleton parsing is a component of text-to-SQL models that helps generate the structure of an SQL query based on a natural language question. It focuses on generating the pure skeleton of an SQL query (i.e., SQL keywords).