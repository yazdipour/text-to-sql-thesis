\subsection{Terminology}

Here is an updated list of key terminology and vocabulary that you may need to know before studying Text-to-SQL language models:

% \subsubsection{SQL (Structured Query Language)} The standard, widely used programming language designed to manage relational databases, which enables users to store, retrieve and manipulate data.
\subsubsection{Natural Language Processing (NLP)}

The field of study focused on the interaction between human language and computers, which ranges from understanding spoken language to generating natural language text.

\subsubsection{Tokenization}

The process of breaking up a sentence into individual words or phrases, that is necessary for tasks such as machine translation and text summarization.

\subsubsection{Word2Vec}

X\cite{DBLP:journals/corr/Rong14}

\subsubsection{WordPiece embeddings}

X\cite{DBLP:journals/corr/WuSCLNMKCGMKSJL16}

\subsubsection{Encoder-Decoder Architecture}

A powerful neural network architecture that utilizes an encoder\cite{cho-etal-2014-learning} to transform the input data into a compact and meaningful representation and a decoder to generate the desired output from that representation. This architecture has been widely used in many applications such as language translation, image captioning, and text summarization to produce high-quality results. Furthermore, the encoder-decoder architecture has the advantage of being able to learn complex relationships between input and output, making it a suitable tool for many challenging tasks.

\subsubsection{Transformers}

The architecture introduced in the paper "Attention Is All You Need" by Vaswani et al. in 2017\cite{https://doi.org/10.48550/arxiv.1706.03762}, known as Transformers, is a revolutionary breakthrough in the way sequences of data are processed. By utilizing self-attention mechanisms, the model is able to achieve improved efficiency and accuracy, while also being much simpler to implement and deploy. This makes it particularly appealing for a wide range of applications, from natural language processing to computer vision. Furthermore, due to their scalability, Transformers are able to accommodate large data sets, enabling them to be used to tackle more complex tasks. As such, Transformers are becoming increasingly popular in the field of machine learning and artificial intelligence, with more and more research being done to further explore its capabilities.

There were many excellent works around 2015 on learning word vectors to continuous representations for words where the identity of a word was mapped to a fixed-length vector which ideally encoded some meaning about the word in a continuous space and for a long time.

That has been an essential part of the NLP pipeline, especially for deep learning models where these pre-trained word vectors were used, typically trained using an unsupervised objective, and new models were fed and trained on top of them.

An important paper in 2017 that helped researchers change their way of thinking towards the transfer learning paradigm was the unsupervised sentiment neuron paper from people at OpenAI \cite{DBLP:journals/corr/RadfordJS17}, which essentially showed that by just training a language model on a purely unsupervised objective, the model could learn concepts that were potentially useful for downstream tasks.

In 2018, the NLP community had a couple of super important papers, \\
including the ULMFiT\cite{ELMo}, which took the recipe from semi-supervised sequence learning, added some tweaks, figured out how to get it working better, and got some noble results with a similar pipeline, pre-training a language model, fine-tuning on a downstream task.

And then, ELMo\cite{ELMo} showed that we could get significantly better performance by using a bi-directional language model.

Then GPT1 \cite{Radford2018ImprovingLU} came along, saying that instead of using analyst TM, we can get good performance by using a transformer with a language model.

Finally, in 2018, BERT \cite{devlin-etal-2019-bert} showed that a bi-directional transformer could get outstanding performance, and by the end of 2018, many researchers were convinced that this was the path forward given all of the impressive results that these papers and a few others showed.

Following these researches, there has been a burst of work on transfer learning for NLP, working on various methods, different pre-training ideas, datasets, and different benchmark tasks.

In Google T5, it is tried to use all the new studies in transfer learning and combine the best selection of these studies to achieve state-of-the-art results on many benchmarks covering summarization, classification, question answering, and more.

\subsubsection{Self-attention}

Self-attention\cite{https://doi.org/10.48550/arxiv.1706.03762} is a mechanism used in the transformer architecture that enables the model to identify the significance of different elements of the input sequence, so as to be able to generate an output that is more accurate and effective. This mechanism allows the model to take into account the relationships between different parts of the input sequence and to factor those relationships into its output. Additionally, self-attention allows the model to capture patterns from the input sequence and to use those patterns to generate more meaningful output. It is this combination of factors that makes self-attention such an important tool for deep learning models.

\subsubsection{Pre-training and Fine-tuning}

Pre-training refers to training a model on a large dataset and then fine-tuning it on a smaller dataset for a specific task, which helps to improve the model's performance on the specific task.

\subsubsection{Long Short-Term Memory (LSTM)}

A type of recurrent neural network that has been designed to store information over a longer period of time than traditional neural networks, allowing it to better capture long-term dependencies\cite{Hochreiter1997LongSM}.
This makes it especially well-suited for tasks such as language modeling and text generation, where it can take into account the context of the text in order to generate more accurate outputs.
In addition, LSTM networks are able to identify patterns in the data that would be difficult for traditional networks to capture. This makes them ideal for tasks such as sequence prediction and classification, where they can identify patterns that would otherwise be too subtle for traditional networks to detect.

\subsubsection{Bidirectional Encoder Representations from Transformers(BERT)}

A pre-trained Transformer model that has been trained on a large corpus of text, with the primary aim of pre-training language representations for use in natural language processing tasks\cite{devlin-etal-2019-bert}. This pre-training helps to give BERT a strong understanding of the language structure and helps in faster training times for downstream tasks. BERT can be fine-tuned for various applications, such as Text-to-SQL, where it can provide better performance than non-specialized models. By leveraging the already learned representations from the pre-trained model, BERT is able to quickly adjust to the task at hand, resulting in faster training times.

\subsubsection{SQL Constructs}

The elements of SQL language such as SELECT, FROM, WHERE, JOIN, are used to build queries and retrieve data from a database.

\subsubsection{Evaluation Metrics}

Measures used to evaluate the performance of Text-to-SQL models, such as accuracy, F1-score, and BLEU score, are used to compare different models and determine the best-performing model.

\subsubsection{Baseline Model}

A model that serves as a reference point or starting point for comparison, providing a baseline for performance against which other models can be evaluated.

\subsubsection{Incremental decoding}

A decoding strategy where the model generates a sequence of tokens one at a time, at each step conditioned on the previous tokens, the input, and the context of the sentence. This approach allows for a more dynamic and flexible generation of output, as it takes into account a variety of factors when making decisions about the next token. This strategy also helps the model avoid repeating itself, providing more diverse and unique outputs. Furthermore, incremental decoding helps the model to better capture the nuance of the language as it is able to build upon previous decisions and refine its output as it progresses\cite{huang-mi-2010-efficient}.

\subsubsection{Semantic parsing}

Semantic parsing\cite{krishnamurthy-etal-2017-neural} is an area of natural language processing that involves extracting the meaning or intent from text. One type of Semantic Parsing, Text-to-SQL, involves the conversion of natural language problems into SQL query statements. This is a challenging task, one that requires the use of advanced machine learning and natural language processing algorithms. As such, the research conducted in this field seeks to explore the various solutions and practices that have been employed by researchers in order to effectively tackle this problem. Furthermore, it is also important to note that this problem is not just limited to the conversion of natural language into SQL query statements, as there are other applications of Semantic Parsing that have been explored, such as Natural Language Generation (NLG). Overall, by understanding the various techniques used for Semantic Parsing, we can gain a better understanding of the complexities involved in this task and how best to approach it.