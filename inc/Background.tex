\section{Technical Background}

In this chapter, we provide background information about the technical concepts related to the main topics of this thesis, which focus on natural language understanding and text generation. We focus on early and recent approaches and the terminology needed to understand the basics of this thesis.

The text-to-SQL problem, or \ac{NL2SQL}, is the following: Given a Natural Language Query (NLQ) on a \ac{RDB}, produce a SQL query equivalent to the \ac{NLQ}. It has been a holy grail for the database community for over 30 years to translate user queries into SQL. Several challenges include ambiguity, schema linking, vocabulary gaps, and user errors.

Early approaches to Text-to-SQL relied on rule-based and template-based methods, while recent approaches use neural networks and machine learning techniques. This allows them to handle a wide range of natural language inputs and generate more accurate SQL queries, which we will discuss further.

\subsection{Early Approaches}
Early approaches to Text-to-SQL focused on rule-based methods and template-based methods. These approaches relied on predefined templates and a set of predefined rules to generate SQL queries. These methods were based on the idea that a fixed set of templates and rules could be used to generate SQL queries for a wide range of natural language inputs. However, these methods were limited by their reliance on predefined templates and were not able to handle a wide range of natural language inputs.

\subsubsection{Rule-based methods}
In the case of rule-based methods, a set of predefined rules were used to map the natural language input to the corresponding SQL query. These rules were based on predefined grammar and were used to identify the SQL constructs present in the input text. These methods were able to generate simple SQL queries, but they were not able to handle more complex queries or handle variations in natural language inputs.

Early research in Text-to-SQL includes work by researchers such as Warren and Pereira in 1982\cite{Warren1982AnEE}, who proposed a rule-based method for generating SQL queries from natural language text. Their system used a set of predefined rules to map natural language constructs to SQL constructs and was able to generate simple SQL queries. Another example of a rule-based method is the work by Zelle and Mooney in 1996, who proposed CHILL parser\cite{Zelle1996LearningTP}, a system that used a predefined grammar to identify the SQL constructs present in the input text and generate the corresponding SQL query.

\subsubsection{Template-based methods}
Template-based methods, on the other hand, relied on predefined templates to generate SQL queries. These templates were based on a predefined set of SQL constructs and were used to map the natural language input to the corresponding SQL query. These methods were able to handle a limited set of natural language inputs, but they were not able to handle variations in the input or generate more complex queries. One of the very first systems that used predefined templates to map natural language inputs to SQL queries was able to handle a limited set of natural language inputs.

In summary, early approaches to Text-to-SQL were limited by their reliance on predefined templates and rules, which made them unable to handle a wide range of natural language inputs and generate complex SQL queries. The rule-based and template-based methods were two of the most common early approaches used in Text-to-SQL, each with their own strengths and limitations.
\subsection{Terminology}

Here is an updated list of key terminology and vocabulary that you may need to know before studying Text-to-SQL language models:

% \subsubsection{SQL (Structured Query Language)} The standard, a widely used programming language designed to manage relational databases, enables users to store, retrieve and manipulate data.
% \subsubsection{Natural Language Processing (NLP)}
% The field of study focuses on the interaction between human language and computers, which ranges from understanding spoken language to generating natural language text.

\subsubsection{Pre-training and Fine-tuning}

Pre-training refers to training a model on a large dataset and then fine-tuning it on a smaller dataset for a specific task, which helps to improve the model's performance on the specific task.

\subsubsection{SQL Constructs}

The elements of SQL language such as SELECT, FROM, WHERE, JOIN, are used to build queries and retrieve data from a database.

% \subsubsection{Evaluation Metrics}

% Measures used to evaluate the performance of Text-to-SQL models, such as accuracy, F1-score, and Exact Match score, compare different models and determine the best-performing model.

\subsubsection{Baseline Model}

A model that serves as a reference point or starting point for comparison, providing a baseline for performance against which other models can be evaluated.


% \subsubsection{Encoder-Decoder Architecture}

% A robust neural network architecture that utilizes an encoder\cite{cho-etal-2014-learning} to transform the input data into a compact and meaningful representation and a decoder to generate the desired output from that representation. This architecture has been widely utilized in many applications, such as language translation, image captioning, and text summarization, to produce high-quality results. Furthermore, the encoder-decoder architecture has the advantage of learning complex relationships between input and output, making it a suitable tool for many challenging tasks\cite{kumar2022deep}.


\subsubsection{Self-attention}

Self-attention \cite{https://doi.org/10.48550/arxiv.1706.03762} is a mechanism used in the transformer architecture that allows the model to determine the significance of various components of the input sequence to be able to generate an outcome that is more precise and sufficient. This mechanism allows the model to consider the relationships between different parts of the input sequence and to factor those relationships into its output. Further, self-attention lets the model capture patterns from the input sequence and use those patterns to generate more meaningful output. It is this combination of factors that makes self-attention such an essential tool for deep learning models.

\subsubsection{Incremental decoding}

A decoding strategy where the model generates a sequence of tokens one at a time, at each step conditioned on the previous tokens, the input, and the context of the sentence. This approach allows for a more dynamic and flexible generation of output, as it takes into account a variety of factors when making decisions about the next token. This strategy also helps the model avoid repeating itself, providing more diverse and unique outputs. Furthermore, incremental decoding helps the model to capture the nuance of the language better as it is able to build upon previous decisions and refine its output as it progresses\cite{huang-mi-2010-efficient}.

\subsubsection{Semantic parsing}

Semantic parsing\cite{krishnamurthy-etal-2017-neural} is an area of natural language processing that involves extracting the meaning or intent from text. One class of Semantic Parsing, Text-to-SQL, involves converting natural language problems into SQL query statements. This is a challenging task, one that requires the use of advanced machine learning and natural language processing algorithms. As such, the research conducted in this field seeks to explore the various solutions and practices employed by researchers to tackle this problem effectively. Furthermore, it is also important to note that this problem is not just limited to converting natural language into SQL query statements, as other applications of Semantic Parsing have been explored, such as \ac{NLG}. Overall, by understanding the various techniques used for Semantic Parsing, we can better understand the complexities involved in this task and how best to approach it.


\subsection{Text Processing}

In this section, we will discuss the fundamental elements of the text processing pipeline in natural language processing (NLP) tasks. The pipeline involves several steps, including tokenization, embedding, prediction, and conversion of embeddings back to words. These elements work together to transform raw text into a suitable format for NLP models and generate human-readable output.

\subsubsection{Tokenization}

Tokenization is a fundamental process in natural language processing (NLP) that involves breaking the raw text into individual words, phrases, or other meaningful units called tokens. This step is essential for preparing the text for further processing, as it enables the model to analyze and understand the text at a more granular level, thus simplifying the analysis and processing of the content.
Various tokenization methods can be employed, including rule-based, statistical, and more advanced approaches like subword tokenization and Byte Pair Encoding (BPE). Rule-based methods often use pre-defined rules to separate words, phrases, or sentences, while statistical methods rely on the frequency and distribution of words and characters in the text.
Subword tokenization and Byte Pair Encoding (BPE) are advanced techniques that account for the morphological structure of words, generating tokens based on common subword units. This approach is beneficial for handling out-of-vocabulary words and capturing meaningful information from rare or previously unseen words.
In addition to word-level tokenization, other techniques exist, such as subword-level and character-level tokenization. Subword-level tokenization further divides words into smaller subword units, which can capture more linguistic information and improve the model's ability to handle morphologically rich languages. On the other hand, character-level tokenization breaks text into individual characters, providing even more granularity and enabling the model to learn character-level patterns.

\subsubsection{Embeddings}

WordPiece embeddings\cite{DBLP:journals/corr/WuSCLNMKCGMKSJL16} is a tokenization approach used in natural language processing (NLP) to break down words into smaller units, also known as pieces. It is an extension of the original word2vec parameter learning algorithm and is used to address out-of-vocabulary (OOV) words, which are words that did not appear in the training data.
This technique divides each word into a series of subword units learned during the training phase based on their frequency and consistency within words. These subword units are stored in a shared vocabulary, dubbed the WordPiece vocabulary, and can be used for multiple words.
This system can represent rare or unseen words as a combination of more common subword units, which are more likely to be in the vocabulary. As a result, the model can handle OOV words more efficiently and reduce the vocabulary size, leading to a more economical representation of the language.
In NLP models, words are usually portrayed as dense vectors referred to as word embeddings. WordPiece embeddings extend this representation by breaking words down into subword units and representing each piece as a dense vector. These subword embeddings are then combined to represent the whole word.
The use of WordPiece embeddings has various advantages in NLP models. Firstly, it enables the model to treat OOV words more effectively by representing them as a combination of more common subword units. Secondly, it decreases the vocabulary size, resulting in a more succinct representation of the language. Finally, it enhances the model's capability to learn fine-grained representations of words and their meanings, resulting in improved performance in NLP tasks.

\subsubsubsection{Word2Vec}

Word2Vec\cite{DBLP:journals/corr/Rong14} is a well-known word embedding approach in NLP that encodes words as dense vectors in an unending, high-dimensional area. This technique is designed to capture the significance and context of words, providing an improved representation of words compared to classic one-hot encoding.
The fundamental concept behind Word2Vec is to train a neural network to anticipate the context words about a target word, given the target word. As the model is trained, the weights of the neural network are adjusted in such a way that the dot product of the input layer (representing the target word) and the output layer (representing the context words) closely estimate the probability distribution of the context words given the target word.
Word2Vec can be trained to employ two different algorithms: Continuous Bag-of-Words (CBOW) and Skip-gram. CBOW predicts the target word given the context words, while Skip-gram predicts the context words given the target word. The algorithm selection relies on the particular NLP task and the data available for training.

\subsubsubsection{\ac{GloVe}}

Global Vectors for Word Representation, developed by Pennington et al.\cite{pennington-etal-2014-glove}, is another popular embedding technique. GloVe combines the advantages of both global matrix factorization methods and local context window methods. It learns embeddings by considering the co-occurrence probabilities of words within a corpus, thus capturing the global corpus statistics. GloVe embeddings demonstrate better performance on various NLP tasks, such as semantic similarity and analogy detection, compared to Word2Vec.

\subsubsubsection{\ac{ELMo}}

Embeddings from Language Models, introduced by Peters et al.\cite{ELMo}, is a more advanced approach that generates contextualized word embeddings. Unlike Word2Vec and GloVe, which produce static embeddings for each word, ELMo generates embeddings that are context-dependent. ELMo is based on a bidirectional LSTM language model, which learns different layers of representations for each word, capturing both low-level syntactic features and high-level semantic features. The contextualized nature of ELMo embeddings has proven to significantly improve performance in various downstream NLP tasks.

\subsubsection{Prediction}

Prediction is a critical step in many NLP tasks, such as text classification, named entity recognition, and machine translation. In this step, models utilize embeddings to make predictions or generate output based on the input text. Various techniques can be employed for prediction, including feedforward neural networks, recurrent neural networks (RNNs), long short-term memory (LSTM) networks, gated recurrent units (GRUs), and transformer-based models.

\subsubsubsection{\ac{LSTM}}

A type of recurrent neural network designed to store information over a more extended period than traditional neural networks, allowing it to capture long-term dependencies better \cite{Hochreiter1997LongSM}.
This makes it especially well-suited for tasks such as language modeling and text generation, where it can take into account the context of the text in order to generate more accurate outputs.
In addition, LSTM networks can identify patterns in the data that would be difficult for traditional networks to capture. This makes them ideal for tasks such as sequence prediction and classification, where they can identify patterns that would otherwise be too subtle for traditional networks to detect.

\subsubsection{Conversion of Embeddings Back to Words}

The final step in the text processing pipeline is converting the embeddings back into human-readable text. This process is typically part of the decoding phase in sequence-to-sequence models, where the model takes the embeddings and generates a sequence of words in the target language or format. Several decoding strategies can be used to achieve this, such as greedy search, beam search, and sampling methods like top-K sampling and nucleus sampling. We will discuss these strategies in more detail in the next section.
The architecture introduced in the paper "Attention Is All You Need" by Vaswani et al. in 2017\cite{https://doi.org/10.48550/arxiv.1706.03762}, known as Transformers, is a revolutionary breakthrough in the way sequences of data are processed. By utilizing self-attention mechanisms, the model is able to achieve improved efficiency and accuracy, while also being much simpler to implement and deploy. This makes it particularly appealing for a wide range of applications, from natural language processing to computer vision. Furthermore, due to their scalability, Transformers are able to accommodate large data sets, enabling them to be used to tackle more complex tasks. As such, Transformers are becoming increasingly popular in the field of machine learning and artificial intelligence, with more and more research being done to further explore its capabilities.

There were many excellent works around 2015 on learning word vectors to continuous representations for words where the identity of a word was mapped to a fixed-length vector which ideally encoded some meaning about the word in a continuous space and for a long time.

That has been an essential part of the NLP pipeline, especially for deep learning models where these pre-trained word vectors were used, typically trained using an unsupervised objective, and new models were fed and trained on top of them.

An important paper in 2017 that helped researchers change their way of thinking towards the transfer learning paradigm was the unsupervised sentiment neuron paper from people at OpenAI \cite{DBLP:journals/corr/RadfordJS17}, which essentially showed that by just training a language model on a purely unsupervised objective, the model could learn concepts that were potentially useful for downstream tasks.

In 2018, the NLP community had a couple of super important papers, \\
including the ULMFiT\cite{ELMo}, which took the recipe from semi-supervised sequence learning, added some tweaks, figured out how to get it working better, and got some noble results with a similar pipeline, pre-training a language model, fine-tuning on a downstream task.

And then, ELMo\cite{ELMo} showed that we could get significantly better performance by using a bi-directional language model.

Then GPT1 \cite{Radford2018ImprovingLU} came along, saying that instead of using analyst TM, we can get good performance by using a transformer with a language model.

Finally, in 2018, BERT \cite{devlin-etal-2019-bert} showed that a bi-directional transformer could get outstanding performance, and by the end of 2018, many researchers were convinced that this was the path forward given all of the impressive results that these papers and a few others showed.

Following these researches, there has been a burst of work on transfer learning for NLP, working on various methods, different pre-training ideas, datasets, and different benchmark tasks.

In Google T5, it is tried to use all the new studies in transfer learning and combine the best selection of these studies to achieve state-of-the-art results on many benchmarks covering summarization, classification, question answering, and more.
\subsection{Learning Techniques}

The advancement of Text-to-SQL research has been driven by various learning techniques that address specific challenges in the field. We provide a comprehensive overview of these learning techniques, focusing on both fully supervised and weakly supervised methods.

\subsubsection{Fully Supervised Learning Techniques}

Fully supervised learning approaches depend on labeled data to train models. We explore various cutting-edge methods proposed to enhance Text-to-SQL generation.

\subsubsection*{Active Learning}

Active learning aims to reduce the labeled data needed for training by selectively identifying the most informative examples. Ni et al.\cite{ni2019merging} developed an active learning framework that uses uncertainty estimation to pinpoint samples that would gain the most from human annotations.

\subsubsection*{Interactive/Imitation Learning}

Interactive or imitation learning concentrates on learning from demonstrations, with a model trying to replicate expert behavior. Yao et al. \cite{yao-etal-2019-model} presented an interactive learning method that integrates user feedback to improve the model's comprehension of intricate SQL queries.

\subsubsection*{Multi-task Learning}

Multi-task learning consists of training a single model on multiple related tasks concurrently. Chang et al. \cite{chen2021leveraging} investigated a multi-task learning framework for Text-to-SQL generation, illustrating that sharing information across tasks can result in enhanced performance.

\subsubsection{Weakly Supervised Learning Techniques}

Weakly supervised learning approaches employ weak or noisy labels for training, often diminishing the need for extensive human annotation.

\subsubsection*{Reinforcement Learning}

Reinforcement learning focuses on learning through trial and error, with models receiving feedback via rewards or penalties. Seq2SQL Zhong et al. \cite{zhong_seq2sql_2017} applied reinforcement learning to Text-to-SQL generation, demonstrating that such a method can effectively learn from weak supervision, for instance, it allowed Seq2SQL to understand different orders of WHERE clauses in a query.

\subsubsection*{Meta-learning and Bayesian Optimization}

Meta-learning entails training models to learn how to learn effectively. Huang et al. \cite{huang-etal-2018-natural} suggested a meta-learning strategy for Text-to-SQL tasks, enabling the model to swiftly adapt to new tasks or domains with limited labeled data.

Agarwal et al. \cite{pmlr-v97-agarwal19e} combined meta-learning and Bayesian optimization for weakly supervised Text-to-SQL tasks. This technique enables models to adapt more efficiently to new tasks while taking advantage of limited supervision, ultimately reducing the necessity for large amounts of labeled data.
\subsection{Recent Approaches}

Recent approaches to Text-to-SQL have focused on using neural networks and machine learning techniques to generate SQL queries. These methods use large amounts of training data to learn the relationship between natural language and SQL and can generate SQL queries for a wide range of inputs. These methods can handle a wide range of natural language inputs and are not limited by predefined templates or rules. Additionally, recent approaches leverage pre-trained models such as \ac{BERT} \cite{devlin-etal-2019-bert}, GPT-2 \cite{radford2019language}, and T5 \cite{raffel_exploring_2020}, which have been pre-trained on a large corpus of text, to fine-tune text-to-SQL tasks, which enables them to understand the natural language inputs better and generate more accurate SQL queries.

One favored strategy is using encoder-decoder architecture, which uses an encoder to encode the natural language input and a decoder to generate the corresponding SQL query. The encoder is a pre-trained language model such as BERT, which is fine-tuned on the task of text-to-SQL, and the decoder is a neural network that generates the SQL query. This architecture effectively generates accurate SQL queries for various natural language inputs.

Another recent approach is using reinforcement learning to generate SQL queries, where a neural network generates a sequence of SQL tokens and is trained using a reward signal based on the quality of the generated query. This approach is adequate for generating more complex SQL queries and handling variations in natural language inputs.

In recent years, the Transformer architecture has significantly impacted natural language processing and machine learning, including in the field of Text-to-SQL. The Transformer architecture, presented in the paper "Attention Is All You Need" by Vaswani in 2017 \cite{https://doi.org/10.48550/arxiv.1706.03762}, is a neural network architecture that uses self-attention mechanisms to process data sequences, such as natural language text.

% One of the key advantages of the Transformer architecture is its ability to handle long-term dependencies in data sequences, making it well-suited for tasks such as natural language understanding and text generation. This has led to the development of pre-trained Transformer models, such as BERT, GPT-2, and T5, that have been trained on a large corpus of text and can be fine-tuned on specific tasks such as Text-to-SQL.

The use of pre-trained Transformer models such as BERT in Text-to-SQL has shown to be effective in improving the performance of the models. The pre-trained models have a good understanding of the natural language, which enables them to understand the input text better and generate more accurate SQL queries.
The Transformer architecture and pre-trained models such as BERT have significantly impacted recent studies in the field of Text-to-SQL. The ability of the Transformer architecture to handle long-term dependencies in sequences of data and the pre-trained models' good understanding of natural language has made it possible to generate more accurate SQL queries for a wide range of natural language inputs.

In outline, recent approaches in Text-to-SQL leverage neural networks and machine learning techniques, such as encoder-decoder architecture and reinforcement learning. These approaches use large amounts of training data and pre-trained models such as BERT to generate accurate SQL queries for a wide range of natural language inputs.