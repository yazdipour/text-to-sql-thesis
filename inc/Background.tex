\section{Technical Background}

In this chapter, we provide background information about the technical concepts related to the main topics of this thesis, which focus on natural language understanding and text generation. We focus on early and recent approaches and the terminology needed to understand the basics of this thesis.

The text-to-SQL problem, or NL2SQL, is defined as the following: Given a Natural Language Query (NLQ) on a Relational Database (RDB), produce a SQL query equivalent to the NLQ. Several challenges include ambiguity, schema linking, vocabulary gaps, and user errors. It has been a holy grail for the database community for over 30 years to translate user queries into SQL.

Early approaches to Text-to-SQL relied on rule-based and template-based methods, while recent approaches use neural networks and machine learning techniques. This allows them to handle a wide range of natural language inputs and generate more accurate SQL queries, which we will discuss further.

\subsection{Early Approaches}

Early approaches to Text-to-SQL focused on rule-based methods and template-based methods. These approaches relied on predefined templates and a set of predefined rules to generate SQL queries. These methods were based on the idea that a fixed set of templates and rules could be used to generate SQL queries for a wide range of natural language inputs. However, these methods were limited by their reliance on predefined templates and were not able to handle a wide range of natural language inputs.

In the case of rule-based methods, a set of predefined rules were used to map the natural language input to the corresponding SQL query. These rules were based on predefined grammar and were used to identify the SQL constructs present in the input text. These methods were able to generate simple SQL queries, but they were not able to handle more complex queries or handle variations in natural language inputs.

Template-based methods, on the other hand, relied on predefined templates to generate SQL queries. These templates were based on a predefined set of SQL constructs and were used to map the natural language input to the corresponding SQL query. These methods were able to handle a limited set of natural language inputs, but they were not able to handle variations in the input or generate more complex queries.

Early research in Text-to-SQL includes work by researchers such as Warren and Pereira in 1994, who proposed a rule-based method for generating SQL queries from natural language text. Their system used a set of predefined rules to map natural language constructs to SQL constructs and was able to generate simple SQL queries.
Another example of a rule-based method is the work by Zelle and Mooney in 1996, who proposed a system that used a predefined grammar to identify the SQL constructs present in the input text and generate the corresponding SQL query.

In the case of template-based methods, one example is the work by Chen and Popescu in 1996, who proposed a template-based method for generating SQL queries. Their system used predefined templates to map natural language inputs to SQL queries and was able to handle a limited set of natural language inputs.
Other early template-based techniques, such as the work of Rau and Wiederhold in 1995, used predefined templates to generate SQL queries. The system uses a set of predefined templates, where each template represents a specific SQL construct, and the system tries to match the natural language input to the suitable template to generate the corresponding SQL query.

In summary, early approaches to Text-to-SQL were limited by their reliance on predefined templates and rules, which made them unable to handle a wide range of natural language inputs and generate complex SQL queries.

\subsection{Recent Approaches}

Recent approaches to Text-to-SQL have focused on using neural networks and machine learning techniques to generate SQL queries. These methods use large amounts of training data to learn the relationship between natural language and SQL and can generate SQL queries for a wide range of inputs. These methods are able to handle a wide range of natural language inputs and are not limited by predefined templates or rules. Additionally, recent approaches leverage pre-trained models such as BERT, GPT-2, and T5, which have been pre-trained on a large corpus of text, to fine-tune text-to-SQL tasks, which enables them to understand the natural language inputs better and generate more accurate SQL queries.

One popular approach is the use of encoder-decoder architecture, which uses an encoder to encode the natural language input and a decoder to generate the corresponding SQL query. The encoder is a pre-trained language model such as BERT, which is fine-tuned on the task of text-to-SQL, and the decoder is a neural network that generates the SQL query. This architecture has been shown to be effective in generating accurate SQL queries for a wide range of natural language inputs.

Another recent approach is the use of reinforcement learning to generate SQL queries, where a neural network generates a sequence of SQL tokens and is trained using a reward signal based on the quality of the generated query. This approach has been shown to be adequate in generating more complex SQL queries and handling variations in natural language inputs.

In recent years, the Transformer architecture has had a significant impact on natural language processing and machine learning, including in the field of Text-to-SQL. The Transformer architecture, presented in the paper "Attention Is All You Need" by Vaswani et al. in 2017, is a neural network architecture that uses self-attention mechanisms to process sequences of data, such as natural language text.

One of the key advantages of the Transformer architecture is its ability to handle long-term dependencies in sequences of data, making it well-suited for tasks such as natural language understanding and text generation. This has led to the development of pre-trained Transformer models, such as BERT, GPT-2, and T5, that have been trained on a large corpus of text and can be fine-tuned on specific tasks such as Text-to-SQL.

The use of pre-trained Transformer models such as BERT in Text-to-SQL has shown to be effective in improving the performance of the models. The pre-trained models have a good understanding of the natural language, which enables them to understand the input text better and generate more accurate SQL queries.
The Transformer architecture and pre-trained models such as BERT have had a significant impact on recent studies in the field of Text-to-SQL. The ability of the Transformer architecture to handle long-term dependencies in sequences of data and the pre-trained models' good understanding of natural language has made it possible to generate more accurate SQL queries for a wide range of natural language inputs.

In summary, recent approaches to Text-to-SQL leverage neural networks and machine learning techniques, such as the use of encoder-decoder architecture and reinforcement learning. These approaches use large amounts of training data and pre-trained models such as BERT to generate accurate SQL queries for a wide range of natural language inputs.