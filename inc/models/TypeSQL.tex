\subsubsection{TypeSQL}
TypeSQL \cite{DBLP:journals/corr/abs-1804-09769}, proposed by Yu et al. (2018), is an enhanced version of SQLNet. It introduces a new training process and uses types obtained from knowledge graphs or table content to help the model better understand the entities and numbers in question. This process helps the model comprehend the semantics of the query and effectively use the context information of the database. In our experiment, we extracted question type info from database content and extended the modules to include ORDER BY and GROUP BY components. This makes TypeSQL the only model incorporating database content, providing a more complete and accurate understanding of the query. With these enhanced features, TypeSQL can better understand the query and produce more accurate results.

% direct copy
In contrast, SQLNet and TypeSQL that utilize SQL structure information to guide the SQL generation process significantly outperform other Seq2Seq models. While they can produce valid queries, however, they are unable to generate nested queries or queries with keywords such as EXCEPT and INTERSECT because they limit possible SQL outputs in some fixed pre-defined SQL structures.